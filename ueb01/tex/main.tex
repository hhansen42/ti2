\documentclass{ti2}

% Dateikodierung ist latin1
\usepackage[latin1]{inputenc}   

\begin{document}

% Nr, Abgabedatum, Gruppenleiter, Gruppenname, Name1...Name4
\Abgabeblatt{1}{xx.xx.2012}{Tutor}{Gruppenname}%
                {Teilnehmer 1}{Teilnehmer 2}%
                {Teilnehmer 3}{Teilnehmer 4}%

\section*{Aufgabe 1}
Zuerst der Quellcode zu unserer L�sung f�r Aufgabe 1. Es ist ganz
toller Quellcode. So ganz richtig toll. Extra-prima, um genau zu sein.

% Diese Umgebung zeigt Source-Code mit Zeilennummern an
% \begin{listing}{Nummer der ersten Zeile}
\begin{listing}{1}
#include <iostream.h>

const int zeilen=6;

int main() {
    int n=zeilen;
    while (n>=0){
        int i=0;
        while (i<n){
            cout << " ";
            i=i+1;
        }
\end{listing}
Nach ein paar Bemerkungen geht der Quellcode hier weiter. Denn er ist
immer noch extra-prima und wir wollen nichts davon auslassen.
% Im Unterschied zu der listing-Umgebung wird hier die
% Numerierung einer vorherigen listing-Umgebung fortgesetzt.
\begin{listingcont}
        i=0;
        while (i<=(zeilen-n)*2){
            cout << "*";
            i=i+1;
        }

        cout << endl;
        n=n-1;
    }

    return 0;
}
\end{listingcont}

%%

\section*{Aufgabe 2}

Bla foo bar fasel. Fnunk. Bla foo bar fasel. T�del�t. Bla foo bar
fasel. Hallo. Bla foo bar fasel. Fnunk. Bla foo bar fasel. Fnunk. Bla
foo bar fasel. Fnunk. Bla foo bar fasel.

\end{document}
