\documentclass{ti2}

% Dateikodierung ist utf8
\usepackage[utf8]{inputenc} 
\usepackage[colorlinks=true,urlcolor=blue,linkcolor=blue]{hyperref}

\begin{document}

% Nr, Abgabedatum, Gruppenleiter, Gruppenname, Name1...Name4
\Abgabeblatt{1}{31.10.2012}{}{}%
                {Hauke Hansen}{Lukas Heinrich}%
                {Florian Kraemer}{}%

\section*{Aufgabe 1}
% Diese Umgebung zeigt Source-Code mit Zeilennummern an
% \begin{listing}{Nummer der ersten Zeile}
\begin{listing}{1}
\end{listing}
\begin{listingcont}

\end{listingcont}

\section*{Aufgabe 3}

\section*{Aufgabe 2}
\subsection*{1.}
Die öffentliche Zugänglichkeit der public-keys stellt kein Sicherheitsproblem dar, da SSH auf einem asymmetrischen Verschlüsselungsverfahren beruht. Das zugrundeliegende Prinzip beruht dann auf mathematischen Einwegfunktionen, sprich Funktionen die (mittels public-key) einfach zu berechnen sind. Das Ergebnis aber dann zu invertieren, sprich zu entschlüsseln, ist ohne Kenntnis des private-keys unmöglich. Letzteres ist allerdings eine angenommene unbewiesene Bahauptung für einen endlichen Zeitraum.\footnote{\url{http://de.wikipedia.org/wiki/Einwegfunktion}}\footnote{\url{http://de.wikipedia.org/wiki/Asymmetrisches_Kryptosystem}}


\subsection*{2.}
Das Senden eines Passwortes über einen ungesicherten Kanal könnte leicht abgehört werden, und würde einen Angreifer direkt in Besitz unverschlüsselter Zugangsdaten bringen.
\end{document}
